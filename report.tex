% !TeX program = lualatex 
\documentclass[logo]{mlai-report}

\usepackage[backend=biber,style=authoryear]{biblatex} 
\renewcommand*{\nameyeardelim}{\addcomma\space}

%Use marginnotes to display the authors of the sections
\usepackage{marginnote}
\renewcommand*{\marginfont}{\normalcolor\normalfont\small} 


\usepackage{csquotes} 

%Nice and easy references
\usepackage{varioref}
\usepackage{cleveref}
 

% Please change this according to your situation
\title{The MLAI Report Template} 
\author{Philip J. Fry \and Turanga Leela \and Bender Bending Rodriguez}
\date{\today} 

\course{MA-INF 4209}{Selected Papers from Mars University}{}
\term{Winter 3000}
\supervisor{Prof. Dr. Hubert J. Farnsworth \and Dr. John A. Zoidberg} 

\addbibresource{references.bib} 
\usepackage{blindtext} 

\begin{document}
	\maketitle
	
	\begin{abstract}\marginnote{Leela}
		This is at the same time the official template for lab and seminar reports within the MLAI group at the University of Bonn and a documentation of its usage.
		You can start your lab/seminar report or Projektgruppenbericht by modifying \texttt{report.tex}.
		You must not change \texttt{mlai-report.cls} and must not fiddle with the page margins or other spacing.
		The body of this document showcases how to use our document class.
	\end{abstract}
	

	\section{Introduction}\marginnote{Fry}
	
	\begin{equation}
	\int_{-\infty}^\infty \mathrm{e}^{-\frac{x^2}{2}} \; \mathrm{d}x 
	\end{equation}
	
	\enquote{Don't panic} \parencite{adams1979} \blindtext
	
	\subsection{Related Work}\marginnote{Rodriguez}
	
	\section{Preliminaries} 

	\begin{figure}[tp] 
		\centering
		\includegraphics[height=4cm]{example-image-a} 
		\caption{Schubidu} 
		\label{fig:example-image} 
	\end{figure}

	\section{Setup} 
	
	\subsection{Requirements} 
	
	\subsection{Class Options} 
	
	\subsection{Language} 
	
	\subsection{Title and Author}
	
	\subsection{Course, Term, and Supervisor} 
	
	\section{Feature Reel} 
	
	\subsection{Floating Environments} 
	
	\subsection{Algorithms}
	
	\subsection{Theorem-like Environments} 
	
	\subsection{List Environments} 
	
	\subsection{Citations} 
	
	\section{Useful Additions} 
	
	\section{Concluding Remarks}
	
	\printbibliography
	
	\appendix
	
	\section{Known Issues} 
\end{document}
